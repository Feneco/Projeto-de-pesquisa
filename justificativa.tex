\chapter{Tema, Justificativa e Objetivos}

\section*{Tema: Dificuldades no aprendizado durante o ensino fundamental II em crianças com TDAH}

\section*{Problema: Como a computação pode ajudar no aprendizado?}

\section{Justificativa}

Com o avanço da tecnologia é cada vez mais comum vermos o uso dela em diversos aspectos de nosso dia a dia, consequentemente há um aumento na facilidade ao acesso da tecnologia.

É comum professores se depararem com alunos com deficiências de aprendizagem, e para alcançar estes alunos, é necessário evoluir nas metodologias de ensino, de forma que possa ser proporcionada uma aprendizagem efetiva destas crianças e adolescentes. A geração atual de alunos, já nasceu em um mundo onde a tecnologia digital é utilizada em praticamente tudo do dia a dia, o uso correto destas é de grande auxílio para que professores possam alcançar estes alunos.

Trazendo tecnologias que estão presentes estes alunos já possuem afinidade, é possível torná-lo mais receptivo ao aprendizado, atraindo sua atenção, promovendo aprendizado coletivo, adequando o conteúdo de acordo com as necessidades especiais deles.

\section{Objetivos}

\subsection{Objetivo geral:} Analisar como a computação pode auxiliar o aprendizado dos alunos com TDAH do Ensino Fundamental II

\subsection{Objetivos específicos:}

\begin{itemize}
\item Identificar dificuldades que podem ser mitigadas
\item Ferramentas que podem ajudar no aprendizado
\item Pesquisa experimental sobre a eficiência de ferramentas existentes
\end{itemize}

\section{Metodologia}
Para encontrar as possibilidades que a computação pode oferecer para auxiliar ou mitigar dificuldades em crianças com TDAH, uma breve pesquisa bibliográfica inicial sobre o assunto revela que a computação pode estimular mais sentidos que uma aula tradicional na escola, desta forma retendo um foco maior do estudante. Há possibilidade de existirem outras dificuldades em áreas específicas do ensino (como matemática, escrita, entre outras) que ferramentas computacionais como jogos digitais, ou elementos de software podem mitigar, portanto, o início da pesquisa será em formular hipóteses sobre elementos de computação que estimulem o aluno no contexto de aprendizado perante aspectos que podem ser auxiliados.

Buscando ferramentas que apresentem os aspectos desejados, é feita uma lista com aquelas que têm possibilidade de avançar as hipóteses. Tais ferramentas podem ser encontradas em meios científicos e podem ou não serem específicas para ensino. Quando não forem voltadas para o ensino, certos aspectos do software podem ser utilizados quando forem do interesse (como interface de usuário, gráficos ou estímulo ao aluno).

Em seguida, utilizando as ferramentas selecionadas, executa-se um estudo experimental com alunos do ensino fundamental II que tenham o distúrbio (visando as diferenças entre os sintomas que têm) para então analisar e qualificar as hipóteses previamente formuladas concluindo a pesquisa.
