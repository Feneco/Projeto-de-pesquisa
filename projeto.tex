\documentclass[article,12pt,openright,twoside,a4paper,brazil]{abntex2}
\usepackage[T1]{fontenc}		% seleção de códigos de fonte.
\usepackage[utf8]{inputenc}		% determina a codificação utiizada (conversão automática dos acentos)
\usepackage{lmodern}			% Usa a fonte Latin Modern
\usepackage[brazil]{babel}		% idiomas
\usepackage{hyperref}  			% controla a formação do índice
\usepackage{hyphenat}           % Regras de hifenação
\hyphenation{mate-mática recu-perar}

% Suprimir erro de falta de \foreigntigle: https://github.com/abntex/abntex2/issues/218
\ifthenelse{\equal{\ABNTEXisarticle}{true}}{%
\renewcommand{\maketitlehookb}{}
}{}


%##############################################################################
\title{\textbf{Projeto de pesquisa em computação}}
\author{Wagner Clemente Coelho Batalha \and Gabriel Viana Tomm}
\date{\today}
\instituicao{ UNB - Universidade de Brasília }
\preambulo{Projeto de pesquisa da matéria de MPLIC no semestre 1/2022}


\makeatletter
\hypersetup{
     	%pagebackref=true,
		pdftitle={\@title},
		pdfauthor={\@author},
    	pdfsubject={Projeto de pesquisa produzido na matéria MPLIC},
	    pdfcreator={LaTeX with abnTeX2},
		pdfkeywords={atigo científico}{unb}{Métodos produção Licenciatura},
		colorlinks=true,
    	linkcolor=blue,
    	citecolor=blue,
    	filecolor=magenta,
		urlcolor=blue,
		bookmarksdepth=4
}
\makeatother

% Criar index
\makeindex

% O tamanho do parágrafo é dado por:
\setlength{\parindent}{1.3cm}

% Controle do espaçamento entre um parágrafo e outro:
\setlength{\parskip}{0.2cm}  % tente também \onelineskip

% Espaçamento simples
\SingleSpacing

%##############################################################################
\begin{document}

% Caso queira, o comando \maketitle imprime apenas o título
% \maketitle
% Pode-se usar os comando abaixo pra produzir uma capinha bonitinha
\imprimircapa
\imprimirfolhaderosto

% Sumário
\pdfbookmark[0]{\contentsname}{toc}
\tableofcontents*
\cleardoublepage

% Seleciona o idioma do documento (conforme pacotes do babel)
\selectlanguage{brazil}

% Retira espaço extra obsoleto entre as frases.
\frenchspacing


\textual

% Eu estou escrevendo os textos separando cada parte em um arquivo tex própio e
% depois inserindo neste documento com o \input{} para facilidade de
% leitura, escrita do texto e organização em geral.
% Pode apagar esse texto lorem ipsum para colocar o resumo

\chapter{Introdução}

\lipsum[1-8]
\section{Justificativa}

Com o avanço da tecnologia é cada vez mais comum vermos o uso dela em diversos aspectos de nosso dia a dia, consequentemente há um aumento na facilidade ao acesso da tecnologia. 

É comum professores se depararem com alunos com deficiências de aprendizagem, e para alcançar estes alunos, é necessário evoluir nas metodologias de ensino, de forma que possa ser proporcionada uma aprendizagem efetiva destas crianças e adolescentes. A geração atual de alunos, já nasceu em um mundo onde a tecnologia digital é utilizada em praticamente tudo do dia a dia, o uso correto destas é de grande auxílio para que professores possam alcançar estes alunos. 

Trazendo tecnologias que estão presentes estes alunos já possuem afinidade, é possível torná-lo mais receptivo ao aprendizado, atraindo sua atenção, promovendo aprendizado coletivo, adequando o conteúdo de acordo com as necessidades especiais deles. 

\chapter{}

\subsection{Pesquisa experimental\label{experimental}}
Com planejamento rigoroso, seleciona as variáveis de estudo e seus parâmetros que podem lhe influenciar, detectando respostas significantes. Executada em laboratório ou em campo, onde as condições são manipuladas para ficarem adequadas.

\subsection{Pesquisa bibliográfica\label{bibliografica}}
Pesquisa a partir de documentação científica já existente. Toda pesquisa se inicia com uma busca bibliográfica

% TODO: Resenhas

\chapter{Revisão de literatura}

Foram pesquisados diversos artigos e trabalhos relacionados a TDAH, computação, ensino, e mistura dos três. Os mais relevantes foram escolhidos para as resenhas neste capítulo.

\section{Uma Abordagem Sobre A Aplicação De Jogos Digitais Como Tecnologia Assistiva Para Crianças Com Tdah No 
Processo Da Aprendizagem}
\section*{\cite{claudia}}

O artigo consiste em uma pesquisa de investigação do TDAH utilizando tecnologia assistiva através de jogos digitais. Tem objetivo próximo ao deste projeto de pesquisa, com a diferença que busca soluções utilizando jogos, ao invés de softwares em geral.

Destaca importância da atenção para o aprendizado e da necessidade de desenvolver formas de reduzir a dispersão causada pelo TDAH.

\begin{quote}
Partindo dessas premissas, pode-se dizer que, é fundamental despertar a atenção do aluno, para que possa se focar, aprender, através de concentração e, para chegar a este objetivo, torna-se necessário a utilização de novas metodologias que sejam capazes de oportunizar a esses alunos, o aprendizado. \cite{claudia}
\end{quote}
    

\postextual

\end{document}