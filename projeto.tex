%% abtex2-modelo-projeto-pesquisa.tex, v<VERSION> laurocesar
%% Copyright 2012-<COPYRIGHT_YEAR> by abnTeX2 group at http://www.abntex.net.br/ 
%%
%% This work may be distributed and/or modified under the
%% conditions of the LaTeX Project Public License, either version 1.3
%% of this license or (at your option) any later version.
%% The latest version of this license is in
%%   http://www.latex-project.org/lppl.txt
%% and version 1.3 or later is part of all distributions of LaTeX
%% version 2005/12/01 or later.
%%
%% This work has the LPPL maintenance status `maintained'.
%% 
%% The Current Maintainer of this work is the abnTeX2 team, led
%% by Lauro César Araujo. Further information are available on 
%% http://www.abntex.net.br/
%%
%% This work consists of the files abntex2-modelo-projeto-pesquisa.tex
%% and abntex2-modelo-references.bib
%%

% ------------------------------------------------------------------------
% ------------------------------------------------------------------------
% abnTeX2: Modelo de Projeto de pesquisa em conformidade com 
% ABNT NBR 15287:2011 Informação e documentação - Projeto de pesquisa -
% Apresentação 
% ------------------------------------------------------------------------ 
% ------------------------------------------------------------------------

\documentclass[
	% -- opções da classe memoir --
	%article,
	12pt,				% tamanho da fonte
	openright,			% capítulos começam em pág ímpar (insere página vazia caso preciso)
	twoside,			% para impressão em recto e verso. Oposto a oneside
	a4paper,			% tamanho do papel. 
	% -- opções da classe abntex2 --
	%chapter=TITLE,		% títulos de capítulos convertidos em letras maiúsculas
	%section=TITLE,		% títulos de seções convertidos em letras maiúsculas
	%subsection=TITLE,	% títulos de subseções convertidos em letras maiúsculas
	%subsubsection=TITLE,% títulos de subsubseções convertidos em letras maiúsculas
	% -- opções do pacote babel --
	english,			% idioma adicional para hifenização
	brazil,				% o último idioma é o principal do documento
	]{abntex2}

% ---
% PACOTES
% ---

% ---
% Pacotes fundamentais 
% ---
\usepackage{lmodern}			% Usa a fonte Latin Modern
\usepackage[T1]{fontenc}		% Selecao de codigos de fonte.
\usepackage[utf8]{inputenc}		% Codificacao do documento (conversão automática dos acentos)
\usepackage{indentfirst}		% Indenta o primeiro parágrafo de cada seção.
\usepackage{color}				% Controle das cores
\usepackage{graphicx}			% Inclusão de gráficos
\usepackage{microtype} 			% para melhorias de justificação
% ---

% ---
% Pacotes adicionais, usados apenas no âmbito do Modelo Canônico do abnteX2
% ---
\usepackage{lipsum}				% para geração de dummy text
% ---

% ---
% Pacotes de citações
% ---
\usepackage[brazilian,hyperpageref]{backref}	 % Paginas com as citações na bibl
\usepackage[alf]{abntex2cite}	% Citações padrão ABNT

% --- 
% CONFIGURAÇÕES DE PACOTES
% --- 

% ---
% Configurações do pacote backref
% Usado sem a opção hyperpageref de backref
\renewcommand{\backrefpagesname}{Citado na(s) página(s):~}
% Texto padrão antes do número das páginas
\renewcommand{\backref}{}
% Define os textos da citação
\renewcommand*{\backrefalt}[4]{
	\ifcase #1 %
		Nenhuma citação no texto.%
	\or
		Citado na página #2.%
	\else
		Citado #1 vezes nas páginas #2.%
	\fi}%
% ---

% ---
% Informações de dados para CAPA e FOLHA DE ROSTO
% ---
\titulo{\textbf{Computação para educação básica de crianças com TDAH}}
\autor{Wagner Clemente Coelho Batalha \and Gabriel Viana Tomm}
\local{Brasil}
\data{\today}
\instituicao{%
  Universidade de Brasília - UNB
  \par
  Departamento de Ciência da Computação - CIC}
\tipotrabalho{Projeto de pesquisa}
% O preambulo deve conter o tipo do trabalho, o objetivo, 
% o nome da instituição e a área de concentração 
\preambulo{Projeto de pesquisa sobre educação básica adaptado para crianças com TDAH usando computação.}
% ---

% ---
% Configurações de aparência do PDF final

% alterando o aspecto da cor azul
\definecolor{blue}{RGB}{41,5,195}

% informações do PDF
\makeatletter
\hypersetup{
     	%pagebackref=true,
		pdftitle={\@title}, 
		pdfauthor={\@author},
    	pdfsubject={\imprimirpreambulo},
	    pdfcreator={LaTeX with abnTeX2},
		pdfkeywords={tdah}{educacao}{computacao}{projeto de pesquisa}, 
		colorlinks=true,       		% false: boxed links; true: colored links
    	linkcolor=blue,          	% color of internal links
    	citecolor=blue,        		% color of links to bibliography
    	filecolor=magenta,      		% color of file links
		urlcolor=blue,
		bookmarksdepth=4
}
\makeatother
% --- 

% --- 
% Espaçamentos entre linhas e parágrafos 
% --- 

% O tamanho do parágrafo é dado por:
\setlength{\parindent}{1.3cm}

% Controle do espaçamento entre um parágrafo e outro:
\setlength{\parskip}{0.2cm}  % tente também \onelineskip

% ---
% compila o indice
% ---
\makeindex
% ---

% ----
% Início do documento
% ----
\begin{document}

% Seleciona o idioma do documento (conforme pacotes do babel)
%\selectlanguage{english}
\selectlanguage{brazil}

% Retira espaço extra obsoleto entre as frases.
\frenchspacing 

% ----------------------------------------------------------
% ELEMENTOS PRÉ-TEXTUAIS
% ----------------------------------------------------------
% \pretextual

% ---
% Capa
% ---
\imprimircapa
% ---

% ---
% Folha de rosto
% ---
\imprimirfolhaderosto
% ---

% ---
% NOTA DA ABNT NBR 15287:2011, p. 4:
%  ``Se exigido pela entidade, apresentar os dados curriculares do autor em
%     folha ou página distinta após a folha de rosto.''
% ---

% ---
% inserir o sumario
% ---
\pdfbookmark[0]{\contentsname}{toc}
\tableofcontents*
\cleardoublepage
% ---


% ----------------------------------------------------------
% ELEMENTOS TEXTUAIS
% ----------------------------------------------------------
\textual

% Eu estou escrevendo os textos separando cada parte em um arquivo tex própio e
% depois inserindo neste documento com o \input{} para facilidade de
% leitura, escrita do texto e organização em geral.
% Pode apagar esse texto lorem ipsum para colocar o resumo

\chapter{Introdução}

% 1 - Explicar o porquê de pesquisar soluções para aprendizado com TDAH
A importância da educação no homem é primordial para a vida em sociedade que deve ser garantido a todos. Sabendo que a educação é um dos direitos universais no Brasil e que há diversos tipos de dificuldades e deficiências de aprendizagem, a legislação prevê educação especial para aqueles que necessitam de atendimento especializado, portanto, é possível desenvolver  pesquisas voltadas a métodos e tecnologias de ensino para aperfeiçoar e atender este tipo de atendimento.

% 2 - Introduzir o TDAH e suas dificuldades
Deficiências e dificuldades de aprendizado podem tomar as mais diversas formas. Neste estudo limitamos o escopo apenas em busca de resultados em mitigação de dificuldades criadas pelo TDAH. O Transtorno do Déficit de Atenção ou Hiperatividade se apresenta 3 formas: O tipo desatento, O tipo hiperativo e a combinação dos dois. Cada tipo merece atenção separada e todos apresentam sintomas que dificultam o aprendizado \cite{okuyama}, mas que  não interferem a parte intelectual do afetado (Diferentemente de uma deficiência de aprendizado), portanto há um potencial de adaptar o método de ensino, ou de buscar novos meios com tecnologia.

% 3 - Introduzir o crescente uso de tecnologia e seu potencial uso na mitigação
Tendo em vista a crescente adoção de tecnologia pela sociedade, a percepção do usuário e as possibilidades da computação oferecem uma linha de pesquisa em muitas facetas da educação. Principalmente após a pandemia de Covid-19 em 2020 ocasionando em adoção generalizada do ensino à longa distância, houve um aumento de pesquisa científica relacionada a área de educação virtual, onde antes não havia. 

Uma das principais vantagens da computação no ensino é a interação e estimulação intelectual com o usuário ou potencial limitação de distrações em um ambiente virtual controlado. Adaptando meios e formas podem ser encontradas soluções que mitiguem dificuldades de aprendizado naqueles com TDAH.

% 4 - Objetivos gerais
Como objetivo geral desta pesquisa, busca-se formas de garantir o direito de aprendizado a aqueles que necessitam de atendimento especial no ensino fundamental II, por meio de identificação de dificuldades que podem ser mitigadas utilizando computação, encontrar ferramentas já existentes que podem auxiliar no aprendizado e então realizar experimentação com estas ferramentas, estudando eficiência de aprendizado

\section{Tema, Justificativa e Objetivos}

\subsection*{Tema: Educação básica de crianças com TDAH}

\subsection*{Problema: Como a computação pode ajudar no aprendizado?}

\subsection{Justificativa}

Com o avanço da tecnologia é cada vez mais comum vermos o uso dela em diversos aspectos de nosso dia a dia, consequentemente há um aumento na facilidade ao acesso da tecnologia.

É comum professores se depararem com alunos com deficiências de aprendizagem, e para alcançar estes alunos, é necessário evoluir nas metodologias de ensino, de forma que possa ser proporcionada uma aprendizagem efetiva destas crianças e adolescentes. A geração atual de alunos, já nasceu em um mundo onde a tecnologia digital é utilizada em praticamente tudo do dia a dia, o uso correto destas é de grande auxílio para que professores possam alcançar estes alunos.

Trazendo tecnologias que estão presentes estes alunos já possuem afinidade, é possível torná-lo mais receptivo ao aprendizado, atraindo sua atenção, promovendo aprendizado coletivo, adequando o conteúdo de acordo com as necessidades especiais deles.

\subsection{Objetivos}

\subsubsection{Objetivo geral:} Analisar como a computação pode auxiliar o aprendizado dos alunos com TDAH do Ensino Fundamental II

\subsubsection{Objetivos específicos:}

\begin{itemize}
\item Identificar dificuldades que podem ser mitigadas
\item Ferramentas que podem ajudar no aprendizado
\item Pesquisa experimental sobre a eficiência de ferramentas existentes
\end{itemize}

\subsection{Metodologia}
Para encontrar as possibilidades que a computação pode oferecer para auxiliar ou mitigar dificuldades em crianças com TDAH, uma breve pesquisa bibliográfica inicial sobre o assunto revela que a computação pode estimular mais sentidos que uma aula tradicional na escola, desta forma retendo um foco maior do estudante. Há possibilidade de existirem outras dificuldades em áreas específicas do ensino (como matemática, escrita, entre outras) que ferramentas computacionais como jogos digitais, ou elementos de software podem mitigar, portanto, o início da pesquisa será em formular hipóteses sobre elementos de computação que estimulem o aluno no contexto de aprendizado perante aspectos que podem ser auxiliados.

Buscando ferramentas que apresentem os aspectos desejados, é feita uma lista com aquelas que têm possibilidade de avançar as hipóteses. Tais ferramentas podem ser encontradas em meios científicos e podem ou não serem específicas para ensino. Quando não forem voltadas para o ensino, certos aspectos do software podem ser utilizados quando forem do interesse (como interface de usuário, gráficos ou estímulo ao aluno).

Em seguida, utilizando as ferramentas selecionadas, executa-se um estudo experimental com alunos do ensino fundamental II que tenham o distúrbio (visando as diferenças entre os sintomas que têm) para então analisar e qualificar as hipóteses previamente formuladas concluindo a pesquisa.

\chapter{Resenha da unidade 2 de métodos de pesquisa}

Como este trabalho é um projeto de pesquisa, foi utilizado o livro \cite{met_pesquisa} como base de instrução para a montagem deste projeto. O segundo capítulo contêm informações importantes sobre as características de projetos de pesquisa, então foi feito uma resenha dele.

\section{Abordagens}
\subsection{Qualitativa}

A abordagem de uma pesquisa pode ser qualitativa ou quantitativa. O objetivo é produzir novas informações.

Abordagens qualitativas caracterizam fatores do tipo "humano": a realidade que não pode ser quantificada, sem dar valores, notas ou envolver cálculos numéricos, trabalhando com emoções, opiniões, significados, crenças, relacionamentos entre outras.

O pesquisador se preocupa em explicar os porquês das coisas sem considerar métricas numéricas já que o objeto de estudo em questão não conseguir ser mensurado e como consequência, a pesquisa passa pelo julgamento do próprio pesquisador assim recebendo seus preconceitos ou crenças do trabalho, aproximando-o a pesquisa.

\subsection{Quantitativa}
Os resultados da pesquisa quantitativa já envolvem números, grupos ou classificações determinadas. São pesquisas objetivas que tentam retratar o todo a partir de uma amostra razoavelmente grande, considerando a realidade como algo que pode ser descrito a partir de dados.

O pesquisador tem um olhar externo a aquilo que está sendo estudado, muitas vezes sem tentar se integrar a ou interpretar subjetivamente.

\subsection{Comparação}
Ambas qualitativa e quantitativa tem pontos fracos e fortes que se complementam,     tornando a integração entre as duas essencial para a ciência.


\section{Naturezas}
A natureza da pesquisa são as consequências diretas da pesquisa. Uma pesquisa do tipo básico traz conhecimentos novos, que podem ser pontos de partida para outras pesquisas enquanto uma pesquisa aplicada gera conhecimentos práticos para solução de problemas.

Enquanto a pesquisa básica pode ser universal, a pesquisa aplicada se torna local ao problema que está solucionando: pode ser que nem todas as situações a solução funcione.


\section{Objetivos}
\subsection{Pesquisas exploratórias}
São pesquisas que envolvem entrevistas, levantamento bibliográfico, análise de exemplos para criar hipóteses do problema(ou criar familiaridade).

Procedimentos:
\begin{itemize}
    \item Pesquisa bibliográfica
    \item Estudo de caso
\end{itemize}

\subsection{Pesquisa descritiva}
Descrevem a realidade/fatos/fenômenos. Podem ter problemas de precisão, com coleta de dados ou com exame crítico inadequado.

Procedimentos
\begin{itemize}
    \item Análise documental
    \item Estudo de caso
    \item Pesquisa ex-post-facto
\end{itemize}

\subsection{Pesquisa explicativa}
Buscam os fatores que levam às ocorrências dos fatos, fornecendo explicações aos resultados de outras pesquisas, muitas vezes pesquisas descritivas.

Procedimentos:
\begin{itemize}
    \item Experimentais
    \item Pesquisa ex-post-facto
\end{itemize}


\section{Procedimentos}
O procedimento traz a realidade ao papel, recursivamente refinando e investigando todos os aspectos da realidade, comprovando hipóteses, descrevendo ou explorando, sendo um trabalho interminável de ciência.

\subsection{Pesquisa experimental\label{experimental}}
Com planejamento rigoroso, seleciona as variáveis de estudo e seus parâmetros que podem lhe influenciar, detectando respostas significantes. Executada em laboratório ou em campo, onde as condições são manipuladas para ficarem adequadas.

\subsection{Pesquisa bibliográfica\label{bibliografica}}
Pesquisa a partir de documentação científica já existente. Toda pesquisa se inicia com uma busca bibliográfica

\subsection{Pesquisa documental}
Semelhante a pesquisa bibliográfica, mas não se limita à documentação científica ao incluir cartas, jornais, filmes, tapeçaria etc. na pesquisa.

\subsection{Pesquisa de campo}
Completando a pesquisa documental e bibliográfica, a pesquisa de campo coleta dados com pessoas

\subsection{Pesquisa Ex-Post-Facto}
Pesquisa relações de causa e efeito em fatos que ocorreram. É feita quando não há possibilidade de manipular variáveis necessárias para estudo, sendo executada após o acontecimento.

\subsection{Pesquisa de levantamento}
Conhecido também como censo, estuda dados de uma população como por exemplo opiniões, atitudes e dados.

\subsection{Pesquisa com survey}
A pesquisa com survey coleta dados diretamente da fonte usando um questionário, mantendo o sigilo do questionado

\subsection{Estudo de caso}
Um estudo focado em apenas um sujeito, um pequeno grupo, uma instituição etc… Podem haver vários estudos de caso paralelos.

\subsection{Pesquisa participante}
O pesquisador se envolve com as pessoas sendo investigadas, se tornando como elas.

\subsection{Pesquisa-ação}
Na pesquisa ação o pesquisador se envolve ativamente de forma cooperativa ou participativa. É fonte de controvérsias

\subsection{Pesquisa etnográfica}
A pesquisa etnográfica estuda um povo, participando, entrevistando e interagindo com os estudados. Entretanto, os pesquisadores não intervêm no ambiente.

\subsection{Pesquisa etnometodológica}
Na pesquisa etnometodológica, revela como as coisas acontecem para um grupo social utilizando diversas formas de documentação

\chapter{Referencial Teórico}

Foram pesquisados diversos artigos e trabalhos relacionados a TDAH, computação, ensino, e mistura dos três. Os mais relevantes foram escolhidos para as resenhas neste capítulo.

Inicialmente foi feito uma resenha do segundo capítulo do livro Métodos de pesquisa \cite{met_pesquisa} onde destaca-se os seguintes seções capítulos que descrevem os principais tipos procedimentos realizados nesta pesquisa:

\begin{quote}
\subsection*{Pesquisa experimental\label{sec:exp}}
Com planejamento rigoroso, seleciona as variáveis de estudo e seus parâmetros que podem lhe influenciar, detectando respostas significantes. Executada em laboratório ou em campo, onde as condições são manipuladas para ficarem adequadas.

\subsection*{Pesquisa bibliográfica\label{sec:bib}}
Pesquisa a partir de documentação científica já existente. Toda pesquisa se inicia com uma busca bibliográfica

\cite{met_pesquisa}
\end{quote}

\section{Levantamento Bibliográfico}

Para fazer uma pesquisa abrangente sobre o assunto, artigos relacionados com os temas da lista abaixo foram pesquisados e então 3 deles foram selecionados para que suas resenhas sejam feitas.

\begin{itemize}
    \item TDAH e dificuldades de educação
	\item Estratégia de ensino para TDAH
    \item Computação no ensino médio
\end{itemize}

\subsection{Os Desafios Do Ensino Aprendizagem Na Modalidade Ead Para Alunos Com TDAH}

O artigo\cite{okuyama} é uma revisão de literatura feita em 2021 sobre TDAH, dificuldades de ensino enfrentados tanto por aulas presenciais quanto aulas á distância. Cita a falta de artigos relacionados á dificuldades de aprendizado dos alunos com TDAH em aulas EAD, citando que possivelmente há muito material em produção, dado a pandemia de covid-19. Também cita que não há muita pesquisa do assunto sobre crianças e adolescentes, já que até então o EAD não era comum entre esta faixa etária. 

O autor destaca:

\begin{quote}
O fato de se ter poucos trabalhos relacionados especificamente com a EAD não trouxe um prejuízo para o estudo, uma vez que alguns autores afirmam que os desafios enfrentados na modalidade EAD são basicamente os mesmos da presencial com adição daqueles gerados pela situação remota das aulas [13,14,15].\cite{okuyama}
\end{quote}

Levanta que tanto professor, escola e sistema de ensino não está preparado para atender estudantes com TDAH seja por acesso a tecnologia, conhecimento do transtorno e métodos didáticos e que políticas inclusivas podem mudar o panorama do trato de alunos com TDAH. 

\begin{quote}
Falando ainda do papel do professor, um dos autores afirma que ele é a figura principal no processo de tornar o ensino aprendizagem mais eficaz em estudantes com TDAH e que seu despreparo pode fazer com que o desafio destes possam ser potencializados independentemente da modalidade presencial ou EAD [20]. Por outro lado, uma política educacional realmente inclusiva, séria e dinâmica, capaz de abandonar o tradicionalismo para se adaptar às novas necessidades, pode mudar o panorama atual do trato com os estudantes com TDAH [21].\cite{okuyama}
\end{quote}

Cita que o ambiente EAD não é o desafio e sim que a desmotivação causada pelas atividades produzidas onde os mecanismos atuais não suprem as necessidades dos alunos. Os alunos com TDAH nem sempre contam com apoio de familiares nos estudos EAD, onde antes nas aulas presenciais os professores supriam. 


\subsection{Jogo Para Auxílio Ao Ensino De Tabuada Principalmente Para Crianças Com TDAH}

O artigo\cite{sanchez} implementa um processo de aprendizagem baseada em jogo (GBL - Game Based Learning) para estudantes do ensino fundamental I. Tem como motivação a facilitação do aprendizado da tabuada para alunos do fundamental e que o jogo produzido em forma de aplicativo de celular pelos autores também funcionasse com alunos não-portadores.

É levantado no artigo diversas vantagens do ensino através de jogo, sendo citados os seguintes pontos:

\begin{quote}
\begin{itemize}
\item \textit{GBL faz uso da ação ao invés da explicação analógica;}
\item \textit{GBL cria motivação e satisfação pessoal;}
\item \textit{GBL reúne vários estilos de aprendizagem;}
\item \textit{GBL reforça o domínio de habilidades;}
\item \textit{GBL proporciona um contexto interativo e desenvolve no aluno a competência da tomada de decisão.}
\end{itemize}
\cite{sanchez}
\end{quote}

Também levanta aspectos da interface, destacando a importância das cores e ilustrações do jogo como formas de motivar a curiosidade. 

O jogo produzido foi então submetido para validação em dois colégios para experimentação, um como grupo de controle e outro como grupo experimental. Foi aplicado no total em 210 alunos do terceiro, quarto e quinto anos.

O artigo conclui que o jogo produzido tem potencial de atrair os alunos com TDAH, e através de análise estatística, conclui-se que o aplicativo não só ajudou os alunos com TDAH, como também reduziu a discrepância de notas entre os alunos com e sem TDAH.


\subsection{Estruturas Metodológicas Direcionadas Ao Ensino De Cinemática Para Educandos Diagnosticados Com Tdah: Utilizando O Modellus Como Interface Interativa Entre A Teoria E A Experimentação}

A dissertação\cite{gomides} propõe um método de ensino onde um programa de computador é utilizado para auxiliar no ensino de física a alunos da rede pública do DF. O programa "Modellus" é um software educativo de matemática, gratuito, que pode produzir gráficos e tabelas de acordo com o que o usuário requisita, provendo uma forma interativa de ensinar.

O autor cita as possibilidades da computação em geral de estimular estudantes com TDAH:

\begin{quote}
Esse sistema foi criado para aumentar a quantidade de estímulos necessários para o aprendizado desse estudante, levando em conta não apenas a quantidade de estímulos, mas sim a qualidade, a duração e a multiplicidade desses estímulos, alternando situações em que o educando é passivo e ativo no processo de ensino/aprendizagem. \cite{gomides}
\end{quote}

o autor usa o software em aulas de física com alunos com TDAH em diversas atividades no ensino da matéria de cinemática. Houve uma melhora gradativa de atenção, concentração e compreensão dos alunos. As estratégias utilizadas tornou a aula mais dinâmica e manteve o aluno com TDAH concentrado. Também cita que os alunos criaram uma intuição maior sobre o tema, ao interagir com o software e modificando variáveis.

Entretanto ele nota que o ensino público do DF tem diversas dificuldades para lidar com o aluno com deficit de atenção, além de falta de apoio por parte de psicólogos e psiquiatras, implicando que mais investimento nesta área por parte do governo seria necessária


\section{Discussão}

Os artigos das seções anterior sugerem principalmente que a produção de material de software voltado para alunos com TDAH esteja unido a um método de ensino e que professores estejam preparados sobre o distúrbio. Podemos destacar também as promissoras vantagens de se estimular o adolescente com TDAH usando métodos computacionais, como aumento da interação pelo aluno, estimulando a mente do aluno, tornando-o ativo, ao invés de passivo, no processo.


% ----------------------------------------------------------
% ELEMENTOS PÓS-TEXTUAIS
% ----------------------------------------------------------
\postextual

% ----------------------------------------------------------
% Referências bibliográficas
% ----------------------------------------------------------
\bibliography{abntex2-modelo-references}

% ----------------------------------------------------------
% Glossário
% ----------------------------------------------------------
%
% Consulte o manual da classe abntex2 para orientações sobre o glossário.
%
%\glossary

% ----------------------------------------------------------
% Apêndices
% ----------------------------------------------------------

% ---
% Inicia os apêndices
% ---
\begin{apendicesenv}

% Imprime uma página indicando o início dos apêndices
\partapendices

\lipsum[55-57]

\end{apendicesenv}
% ---


% ----------------------------------------------------------
% Anexos
% ----------------------------------------------------------

% ---
% Inicia os anexos
% ---
\begin{anexosenv}

% Imprime uma página indicando o início dos anexos
\partanexos

\end{anexosenv}

%---------------------------------------------------------------------
% INDICE REMISSIVO
%---------------------------------------------------------------------

\phantompart

\printindex


\end{document}
