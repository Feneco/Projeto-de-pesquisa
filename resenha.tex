\chapter{Resenha da unidade 2 de métodos de pesquisa}


\section{Abordagem}
\subsection{Qualitativa}

A abordagem de uma pesquisa pode ser qualitativa ou quantitativa. O objetivo é produzir novas informações.

Abordagens qualitativas caracterizam fatores do tipo "humano": a realidade que não pode ser quantificada, sem dar valores, notas ou envolver cálculos numéricos, trabalhando com emoções, opiniões, significados, crenças, relacionamentos entre outras.

O pesquisador se preocupa em explicar os porquês das coisas sem considerar métricas numéricas já que o objeto de estudo em questão não conseguir ser mensurado e como consequência, a pesquisa passa pelo julgamento do próprio pesquisador assim recebendo seus preconceitos ou crenças do trabalho, aproximando-o a pesquisa.

\subsection{Quantitativa}
Os resultados da pesquisa quantitativa já envolvem números, grupos ou classificações determinadas. São pesquisas objetivas que tentam retratar o todo a partir de uma amostra razoavelmente grande, considerando a realidade como algo que pode ser descrito a partir de dados.

O pesquisador tem um olhar externo a aquilo que está sendo estudado, muitas vezes sem tentar se integrar a ou interpretar subjetivamente.

\subsection{Comparação}
Ambas qualitativa e quantitativa tem pontos fracos e fortes que se complementam,     tornando a integração entre as duas essencial para a ciência.


\section{Natureza}
A natureza da pesquisa são as consequências diretas da pesquisa. Uma pesquisa do tipo básico traz conhecimentos novos, que podem ser pontos de partida para outras pesquisas enquanto uma pesquisa aplicada gera conhecimentos práticos para solução de problemas.

Enquanto a pesquisa básica pode ser universal, a pesquisa aplicada se torna local ao problema que está solucionando: pode ser que nem todas as situações a solução funcione.


\section{Objetivos}
\subsection{Pesquisas exploratórias}
São pesquisas que envolvem entrevistas, levantamento bibliográfico, análise de exemplos para criar hipóteses do problema(ou criar familiaridade).

Procedimentos:
\begin{itemize}
    \item Pesquisa bibliográfica
    \item Estudo de caso
\end{itemize}

\subsection{Pesquisa descritiva}
Descrevem a realidade/fatos/fenômenos. Podem ter problemas de precisão, com coleta de dados ou com exame crítico inadequado.

Procedimentos
\begin{itemize}
    \item Análise documental
    \item Estudo de caso
    \item Pesquisa ex-post-facto
\end{itemize}

\subsection{Pesquisa explicativa}
Buscam os fatores que levam às ocorrências dos fatos, fornecendo explicações aos resultados de outras pesquisas, muitas vezes pesquisas descritivas.

Procedimentos:
\begin{itemize}
    \item Experimentais
    \item Pesquisa ex-post-facto
\end{itemize}


\section{Procedimentos}
O procedimento traz a realidade ao papel, recursivamente refinando e investigando todos os aspectos da realidade, comprovando hipóteses, descrevendo ou explorando, sendo um trabalho interminável de ciência.

\subsection{Pesquisa experimental}
Com planejamento rigoroso, seleciona as variáveis de estudo e seus parâmetros que podem lhe influenciar, detectando respostas significantes. Executada em laboratório ou em campo, onde as condições são manipuladas para ficarem adequadas.

\subsection{Pesquisa bibliográfica}
Pesquisa a partir de documentação científica já existente. Toda pesquisa se inicia com uma busca bibliográfica

\subsection{Pesquisa documental}
Semelhante a pesquisa bibliográfica, mas não se limita à documentação científica ao incluir cartas, jornais, filmes, tapeçaria etc. na pesquisa.

\subsection{Pesquisa de campo}
Completando a pesquisa documental e bibliográfica, a pesquisa de campo coleta dados com pessoas

\subsection{Pesquisa Ex-Post-Facto}
Pesquisa relações de causa e efeito em fatos que ocorreram. É feita quando não há possibilidade de manipular variáveis necessárias para estudo, sendo executada após o acontecimento.

\subsection{Pesquisa de levantamento}
Conhecido também como censo, estuda dados de uma população como por exemplo opiniões, atitudes e dados.

\subsection{Pesquisa com survey}
A pesquisa com survey coleta dados diretamente da fonte usando um questionário, mantendo o sigilo do questionado

\subsection{Estudo de caso}
Um estudo focado em apenas um sujeito, um pequeno grupo, uma instituição etc… Podem haver vários estudos de caso paralelos.

\subsection{Pesquisa participante}
O pesquisador se envolve com as pessoas sendo investigadas, se tornando como elas.

\subsection{Pesquisa-ação}
Na pesquisa ação o pesquisador se envolve ativamente de forma cooperativa ou participativa. É fonte de controvérsias

\subsection{Pesquisa etnográfica}
A pesquisa etnográfica estuda um povo, participando, entrevistando e interagindo com os estudados. Entretanto, os pesquisadores não intervêm no ambiente.

\subsection{Pesquisa etnometodológica}
Na pesquisa etnometodológica, revela como as coisas acontecem para um grupo social utilizando diversas formas de documentação
