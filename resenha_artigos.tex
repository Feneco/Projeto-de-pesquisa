% TODO: Resenhas

\chapter{Revisão de literatura}

Foram pesquisados diversos artigos e trabalhos relacionados a TDAH, computação, ensino, e mistura dos três. Os mais relevantes foram escolhidos para as resenhas neste capítulo.

\section{Uma Abordagem Sobre A Aplicação De Jogos Digitais Como Tecnologia Assistiva Para Crianças Com Tdah No 
Processo Da Aprendizagem}
\section*{\cite{claudia}}

O artigo consiste em uma pesquisa de investigação do TDAH utilizando tecnologia assistiva através de jogos digitais. Tem objetivo próximo ao deste projeto de pesquisa, com a diferença que busca soluções utilizando jogos, ao invés de softwares em geral.

Destaca importância da atenção para o aprendizado e da necessidade de desenvolver formas de reduzir a dispersão causada pelo TDAH.

\begin{quote}
Partindo dessas premissas, pode-se dizer que, é fundamental despertar a atenção do aluno, para que possa se focar, aprender, através de concentração e, para chegar a este objetivo, torna-se necessário a utilização de novas metodologias que sejam capazes de oportunizar a esses alunos, o aprendizado. \cite{claudia}
\end{quote}
    