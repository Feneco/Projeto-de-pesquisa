\chapter{Referencial Teórico}

Foram pesquisados diversos artigos e trabalhos relacionados a TDAH, computação, ensino, e mistura dos três. Os mais relevantes foram escolhidos para as resenhas neste capítulo.


\section{Levantamento Bibliográfico}

Para fazer uma pesquisa abrangente sobre o assunto, artigos relacionados com os temas da lista abaixo foram pesquisados e então 3 deles foram selecionados para que suas resenhas sejam feitas.

\begin{itemize}
    \item TDAH e dificuldades de educação
	\item Estratégia de ensino para TDAH
    \item Computação no ensino médio
\end{itemize}

\subsection{Os Desafios Do Ensino Aprendizagem Na Modalidade Ead Para Alunos Com TDAH}

O artigo\cite{okuyama} é uma revisão de literatura feita em 2021 sobre TDAH, dificuldades de ensino enfrentados tanto por aulas presenciais quanto aulas á distância. Cita a falta de artigos relacionados á dificuldades de aprendizado dos alunos com TDAH com aulas EAD, citando que possivelmente há muito material em produção, dado a pandemia de covid-19. Também cita que não há muita pesquisa sobre o assunto sobre crianças e adolescentes, já que até então o EAD não era comum entre esta faixa etária. 

O autor destaca:

\begin{quote}
O fato de se ter poucos trabalhos relacionados especificamente com a EAD não trouxe um prejuízo para o estudo, uma vez que alguns autores afirmam que os desafios enfrentados na modalidade EAD são basicamente os mesmos da presencial com adição daqueles gerados pela situação remota das aulas [13,14,15].
\end{quote}

Levanta que tanto professor, escola e sistema de ensino não está preparado para atender estudantes com TDAH seja por acesso a tecnologia, conhecimento do transtorno e métodos didáticos e que políticas inclusivas podem mudar o panorama do trato de alunos com TDAH. 

\begin{quote}
Falando ainda do papel do professor, um dos autores afirma que ele é a figura principal no processo de tornar o ensino aprendizagem mais eficaz em estudantes com TDAH e que seu despreparo pode fazer com que o desafio destes possam ser potencializados independentemente da modalidade presencial ou EAD [20]. Por outro lado, uma política educacional realmente inclusiva, séria e dinâmica, capaz de abandonar o tradicionalismo para se adaptar às novas necessidades, pode mudar o panorama atual do trato com os estudantes com TDAH [21].
\end{quote}

Cita que o ambiente EAD não é o desafio e sim que a desmotivação causada pelas atividades produzidas onde os mecanismos atuais não suprem as necessidades dos alunos. Os alunos com TDAH nem sempre contam com apoio de familiares nos estudos EAD, onde antes nas aulas presenciais os professores supriam. 


\subsection{Jogo Para Auxílio Ao Ensino De Tabuada Principalmente Para Crianças Com TDAH}

O artigo\cite{sanchez} implementa um processo de aprendizagem baseada em jogo (GBL - Game Based Learning) para estudantes do ensino fundamental I. Tem como motivação a facilitação do aprendizado da tabuada para alunos do fundamental e que o jogo produzido em forma de aplicativo de celular pelos autores também funcionasse com alunos não-portadores.

É levantado no artigo diversas vantagens do ensino através de jogo, sendo citados os seguintes pontos:

\begin{quote}
\begin{itemize}
\item \textit{GBL faz uso da ação ao invés da explicação analógica;}
\item \textit{GBL cria motivação e satisfação pessoal;}
\item \textit{GBL reúne vários estilos de aprendizagem;}
\item \textit{GBL reforça o domínio de habilidades;}
\item \textit{GBL proporciona um contexto interativo e desenvolve no aluno a competência da tomada de decisão.}
\end{itemize}
\end{quote}

Também levanta aspectos da interface, destacando a importância das cores e ilustrações do jogo como formas de motivar a curiosidade. 

O jogo produzido foi então submetido para validação em dois colégios para experimentação, um como grupo de controle e outro como grupo experimental. Foi aplicado no total em 210 alunos do terceiro, quarto e quinto anos.

O artigo conclui que o jogo produzido tem potencial de atrair os alunos com TDAH, e através de análise estatística, conclui-se que o aplicativo não só ajudou os alunos com TDAH, como também reduziu a discrepância de notas entre os alunos com e sem TDAH.

\section{Discussão}

Os artigos da seção anterior sugerem que a produção de material de software voltado para alunos com TDAH esteja unido a um método de ensino e professores que estejam preparados e tenham conhecimento do distúrbio. 

