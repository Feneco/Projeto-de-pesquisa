% TODO: Falta incluir um monte de citações

\chapter{Introdução}

% 1 - Explicar o porquê de pesquisar soluções para aprendizado com TDAH
A importância da educação no homem é primordial para a vida em sociedade que deve ser garantido a todos. Sabendo que a educação é um dos direitos universais no Brasil e que há diversos tipos de dificuldades e deficiências de aprendizagem, a legislação prevê educação especial para aqueles que necessitam de atendimento especializado, portanto, é possível desenvolver  pesquisas voltadas a métodos e tecnologias de ensino para aperfeiçoar e atender este tipo de atendimento.

% 2 - Introduzir o TDAH e suas dificuldades
Deficiências e dificuldades de aprendizado podem tomar as mais diversas formas. Neste estudo limitamos o escopo apenas em busca de resultados em mitigação de dificuldades criadas pelo TDAH. O Transtorno do Déficit de Atenção ou Hiperatividade se apresenta 3 formas: O tipo desatento, O tipo hiperativo e a combinação dos dois. Cada tipo merece atenção separada e todos apresentam sintomas que dificultam o aprendizado \cite{okuyama}, mas que  não interferem a parte intelectual do afetado (Diferentemente de uma deficiência de aprendizado), portanto há um potencial de adaptar o método de ensino, ou de buscar novos meios com tecnologia.

% 3 - Introduzir o crescente uso de tecnologia e seu potencial uso na mitigação
Tendo em vista a crescente adoção de tecnologia pela sociedade, a percepção do usuário e as possibilidades da computação oferecem uma linha de pesquisa em muitas facetas da educação. Principalmente após a pandemia de Covid-19 em 2020 ocasionando em adoção generalizada do ensino à longa distância, houve um aumento de pesquisa científica relacionada a área de educação virtual, onde antes não havia. 

Uma das principais vantagens da computação no ensino é a interação e estimulação intelectual com o usuário ou potencial limitação de distrações em um ambiente virtual controlado. Adaptando meios e formas podem ser encontradas soluções que mitiguem dificuldades de aprendizado naqueles com TDAH.

% 4 - Objetivos gerais
Como objetivo geral desta pesquisa, busca-se formas de garantir o direito de aprendizado a aqueles que necessitam de atendimento especial no ensino fundamental II, por meio de identificação de dificuldades que podem ser mitigadas utilizando computação, encontrar ferramentas já existentes que podem auxiliar no aprendizado e então realizar experimentação com estas ferramentas, estudando eficiência de aprendizado
